% File tacl2018v2.tex
% Sep 20, 2018

% The English content of this file was modified from various *ACL instructions
% by Lillian Lee and Kristina Toutanova
%
% LaTeXery is mostly all adapted from acl2018.sty.

\documentclass[11pt,a4paper]{article}
\usepackage{times,latexsym}
\usepackage{url}
\usepackage[T1]{fontenc}

%% Package options:
%% Short version: "hyperref" and "submission" are the defaults.
%% More verbose version:
%% Most compact command to produce a submission version with hyperref enabled
%%    \usepackage[]{tacl2018v2}
%% Most compact command to produce a "camera-ready" version
%%    \usepackage[acceptedWithA]{tacl2018v2}
%% Most compact command to produce a double-spaced copy-editor's version
%%    \usepackage[acceptedWithA,copyedit]{tacl2018v2}
%
%% If you need to disable hyperref in any of the above settings (see Section
%% "LaTeX files") in the TACL instructions), add ",nohyperref" in the square
%% brackets. (The comma is a delimiter in case there are multiple options specified.)

\usepackage[]{tacl2018v2}




%%%% Material in this block is specific to generating TACL instructions
\usepackage{xspace,mfirstuc,tabulary}
\newcommand{\dateOfLastUpdate}{Sept. 20, 2018}
\newcommand{\styleFileVersion}{tacl2018v2}

\newcommand{\ex}[1]{{\sf #1}}

\newif\iftaclinstructions
\taclinstructionsfalse % AUTHORS: do NOT set this to true
\iftaclinstructions
\renewcommand{\confidential}{}
\renewcommand{\anonsubtext}{(No author info supplied here, for consistency with
TACL-submission anonymization requirements)}
\newcommand{\instr}
\fi

%
\iftaclpubformat % this "if" is set by the choice of options
\newcommand{\taclpaper}{final version\xspace}
\newcommand{\taclpapers}{final versions\xspace}
\newcommand{\Taclpaper}{Final version\xspace}
\newcommand{\Taclpapers}{Final versions\xspace}
\newcommand{\TaclPapers}{Final Versions\xspace}
\else
\newcommand{\taclpaper}{submission\xspace}
\newcommand{\taclpapers}{{\taclpaper}s\xspace}
\newcommand{\Taclpaper}{Submission\xspace}
\newcommand{\Taclpapers}{{\Taclpaper}s\xspace}
\newcommand{\TaclPapers}{Submissions\xspace}
\fi

%%%% End TACL-instructions-specific macro block
%%%%

\title{Formatting Instructions for TACL \TaclPapers \\
(Base files: \styleFileVersion-template.tex \& \styleFileVersion.sty, dated \dateOfLastUpdate)}

% Author information does not appear in the pdf unless the "acceptedWithA" option is given
% See tacl2018v2.sty for other ways to format author information
\author{
 Template Author\Thanks{The {\em actual} contributors to this instruction
 document and corresponding template file are given in Section
 \ref{sec:contributors}.} \\
 Template Affiliation/Address Line 1 \\
 Template Affiliation/Address Line 2 \\
 Template Affiliation/Address Line 2 \\
  {\sf template.email@sampledomain.com} \\
}

\date{}

\begin{document}
\maketitle
\begin{abstract}
  This document contains the formatting requirements for TACL \taclpapers. These
  formatting rules take effect for all \taclpapers received from September 2, 2018
  onwards.
\end{abstract}


\iftaclpubformat
\section{Courtesy warning: Common violations of \taclpaper rules that have
resulted in papers being returned to authors for corrections}

Avoid publication delays by avoiding these.
\begin{enumerate}
\item Violation: incorrect parentheses for in-text citations.  See \S
\ref{sec:in-text-cite} and Table \ref{tab:cite-commands}.
\item Violation: URLs that, when clicked, yield an error such as a 404 or go
to the wrong page.
  \begin{itemize}
     \item Advice: best scholarly practice for referencing URLS would be to also
     include the date last accessed.
  \end{itemize}
\item Violation: non-fulfillment of promise from submission to provide access
instructions (such as a URL) for code or data.
\item Violation: References incorrectly formatted (see \S\ref{sec:references}).
Specifically:
\begin{enumerate}
  \item Violation: initials instead of full first/given names in references.
  \item Violation: missing periods after middle initials.
  \item Violation: incorrect capitalization.  For example, change ``lstm'' to
  LSTM and ``glove'' to GloVe.
  \begin{itemize}
    \item Advice: if using BibTex, apply curly braces within the title field to
    preserve intended capitalization.
  \end{itemize}
  \item Violation: using ``et al.'' in a reference instead of listing all
  authors of a work.
  \begin{itemize}
    \item Advice: List all authors and check accents on author names even when
    dozens of authors are involved.
  \end{itemize}
  \item Violation: not giving a complete arXiv citation number.
  \begin{itemize}
     \item Advice: best scholarly practice would be to give not only the full
     arXiv number, but also the version number, even if one is citing version 1.
  \end{itemize}
  \item Violation: not citing an existing peer-reviewed version in addition to
  or instead of a preprints
    \begin{itemize}
     \item Advice: When preparing the camera-ready, perform an additional check
     of preprints cited to see whether a peer-reviewed version has appeared
     in the meantime.
  \end{itemize}
  \item Violation: book titles do not have the first initial of all main words
  capitalized.
  \item Violation: In a title, not capitalizing the first letter of the first word
  after a colon or similar punctuation mark.
\end{enumerate}
\end{enumerate}
\else
% Submission-specific rules
\section{Courtesy warning: Common violations of \taclpaper rules that have
resulted in desk
rejects}
\begin{enumerate}
  \item Violation: wrong paper format.
  \emph{As of the September 2018 submission round and beyond, TACL requires A4
  format.  This is a change from the prior paper size.}

  \item Violation: main document text smaller than 11pt, or table or figure
  captions in a font smaller than 10pt. See Table \ref{tab:font-table}.

  \item Violation: fewer than seven pages of content or more than ten pages of
  content, {\em including} any appendices. (Exceptions are made for
  re-submissions where a TACL Action Editor explicitly granted a set number of
  extra pages to address reviewer comments.) See
  Section \ref{sec:length}.
  \item Violation: Author-identifying information in the document content or
  embedded in the file itself.
    \begin{itemize}
      \item Advice: Make sure the submitted PDF does \emph{not} embed within it
      any author info: check the document properties before submitting.
      Useful tools include Adobe Reader and {\tt pdfinfo}.
      \item Advice: Check that no URLs (or corresponding websites) inadvertently
      disclose any author information. If software or data is to be distributed,
      mention so in {\em anonymized} fashion.
      \item Advice: Make sure that author names have been omitted
      from the author block. (It's OK to include some sort of anonymous
      placeholder.)
      \item Advice: Do not include acknowledgments in a submission.
      \item Advice: While citation of one's own relevant prior work is as
      encouraged as the citation of any other relevant prior work,
      self-citations should be made in the third, not first, person.
      No citations should be attributed to ``anonymous'' or the like.
      See Section \ref{sec:self-cite}.
    \end{itemize}
\end{enumerate}
\fi


\section{General instructions}

\Taclpapers that do not comply with this document's instructions
risk
\iftaclpubformat
publication delays until the camera-ready is brought into compliance.
\else
rejection without review.
\fi


\Taclpapers should consist of a Portable Document Format (PDF) file formatted
for  \textbf{A4 paper}.\footnote{Prior to the September 2018 submission round, a
different paper size was used.} All necessary fonts should be
included in the  file.

\iftaclpubformat
Note that you will need to provide both a single-spaced and a double-spaced
version; see \S \ref{ssec:layout}.

If you promised to provide code or data at submission, specific instructions for
how to access such resources must be provided.  (Typically, a URL to a stable,
resource-specific site suffices.)

All URLs should be manually checked to verify that they
lead to a valid webpage, and to the site that was intended.
\fi


\section{\LaTeX\ files}

\LaTeX\ files compliant with these instructions are available at the
Author Guidelines section of the
TACL website, \href{https://www.transacl.org/}
{https://www.transacl.org}.\footnote{Last accessed \dateOfLastUpdate.} Use of the
TACL \LaTeX\ files is highly recommended: \emph{MIT Press requires authors to
supply \LaTeX\ source files as part of the publication process}; and
use of the recommended \LaTeX\ files makes conversion to the
required camera-ready format simple.
\iftaclpubformat
Specifically, the conversion can be accomplished by as little as: (1) add
``acceptedWithA'' in the square brackets in the line invoking the TACL package,
like so:
{\footnotesize {\tt {\textbackslash usepackage}[acceptedWithA]\{\styleFileVersion\}}} (2) add author information;
(3) add acknowledgments.
\fi

\subsection{Workarounds for problems with the hyperref package}

The provided files use the hyperref package by default. The TACL files
employs the hyperref package to make clickable links for URLs and other references,
and to make titles of bibliographic items into clickable links to their DOIs
in the generated pdf.\footnote{Indeed, for some versions of acl\_natbib.sty,
DOIs and URLs are not printed out or included in the bibliography in any form
if the hyperref package is not used.}

However, it is known that citations or URLs that cross pages can trigger the
compilation error ``{\tt {\textbackslash}pdfendlink ended up in different nesting
level than {\textbackslash}pdfstartlink}''.  In such cases, you may temporarily
disable the hyperref package and then compile to locate the offending portion of
the tex file; edit to avoid a pagebreak within a link;\footnote{If the problematic
link is part of a reference in the bibliography and you do not wish to
directly edit the corresponding .bbl file, a heavy-handed approach is to
add the line
{\tt \textbackslash interlinepenalty=10000}
just after the line
{\tt \textbackslash sloppy\textbackslash clubpenalty4000\textbackslash widowpenalty4000} in the
``{\tt \textbackslash def\textbackslash thebibliography}'' portion
of the file \styleFileVersion.sty.  This penalty means that LaTex will not allow
individual bibliography items to cross a page break.
}
 and then re-enable the
hyperref package.

To disable it,
add {\tt nohyperref} in the square brackets to pass that option to the TACL package.
For example, change
\iftaclpubformat
\verb+[acceptedWithA]+ in
{\footnotesize {\tt {\textbackslash usepackage}[acceptedWithA]\{\styleFileVersion\}}}
to
\verb+[acceptedWithA,nohyperref]+.
\else
{\tt {\textbackslash usepackage}[]\{\styleFileVersion\}}
to
{\tt {\textbackslash usepackage}[nohyperref]\{\styleFileVersion\}}.
\fi



\section{Length limits}
\label{sec:length}

\iftaclpubformat
Camera-ready documents may consist of as many pages of content as allowed by
the Action Editor in their final acceptance letter.
\else
Submissions may consist of seven to ten (7-10) A4 format (not letter) pages of
content.
\fi

The page limit \emph{includes} any appendices. However, references
\iftaclpubformat
and acknowledgments
\fi
do not count
toward the page limit.

\iftaclpubformat
\else
Exception: Revisions of (b) or (c) submissions may have been allowed
additional pages of content by the prior Action Editor, as specified in their
decision letter.
\fi

\section{Fonts and text size}

Adobe's {Times Roman} font should be used. In \LaTeX2e{} this is accomplished by
putting \verb+\usepackage{times,latexsym}+ in the preamble.\footnote{Should
Times Roman be unavailable to you, use
{Computer Modern Roman} (\LaTeX2e{}'s default).  Note that the latter is about
10\% less dense than Adobe's Times Roman font.}

Font size requirements are listed in Table \ref{tab:font-table}. In addition to
those requirements, the content of figures, tables, equations, etc. must be
of reasonable size and readability.
\begin{table}[t]
\begin{center}
\begin{tabular}{|l|rl|}
\hline \bf Type of Text & \bf Size & \bf Style \\ \hline
paper title & 15 pt & bold \\
\iftaclpubformat
author names & 12 pt & bold \\
author affiliation & 12 pt & \\
\else
\fi
the word ``Abstract'' as header & 12 pt & bold \\
abstract text & 10 pt & \\
section titles & 12 pt & bold \\
document text & 11 pt  &\\
captions & 10 pt & \\
%bibliography & 10 pt & \\
footnotes & 9 pt & \\
\hline
\end{tabular}
\end{center}
\caption{\label{tab:font-table} Font requirements}
\end{table}




\section{Page Layout}
\label{ssec:layout}


The margin dimensions for a page in A4 format (21 cm $\times$ 29.7 cm) are given
in Table \ref{tab:margin-table}.  Start the content of all pages directly under
the top margin.
\iftaclpubformat
\else
(The confidentiality header (\S\ref{sec:ruler-and-header}) for submissions is an
exception.)
\fi


\begin{table}[ht]
\begin{center}
\begin{tabular}{|l|}  \hline
Left and right margins: 2.5 cm \\
Top margin: 2.5 cm \\
Bottom margin: 2.5 cm \\
Column width: 7.7 cm \\
Column height: 24.7 cm \\
Gap between columns: 0.6 cm \\ \hline
\end{tabular}
\end{center}
\caption{\label{tab:margin-table} Margin requirements}
\end{table}


\Taclpapers must be in two-column format.
Allowed exceptions to the two-column format are the title, which must be
centered at the top of the first page;
\iftaclpubformat
the author block containing author names and affiliations and addresses, which
must be centered on the top of the first page and placed after the title;
\else
the  confidentiality header (see \S\ref{sec:ruler-and-header}) on submissions;
\fi
and any full-width figures or tables.

Should the pages be numbered?  Yes, for submissions (to facilitate review); but
no, for camera-readies (page numbers will be added at publication time).

\Taclpapers should be single-spaced.
\iftaclpubformat
But, {\em an additional double-spaced version must also be provided, together with the
single-spaced version, for the use of the copy-editors.}  A double-spaced version can
be created by adding the ``copyedit'' option: Change \verb+[acceptedWithA]+ in
{\footnotesize {\tt {\textbackslash usepackage}[acceptedWithA]\{\styleFileVersion\}}}
to \verb+[acceptedWithA,copyedit]+.
\fi

{Indent} by about 0.4cm when starting a new paragraph that is not the first in a
section or subsection.

\subsection{The confidentiality header and line-number ruler}
\label{sec:ruler-and-header}
\iftaclpubformat
Camera-readies should not include the left- and right-margin line-number rulers
or headers from the submission version.
\else
Each page of the submission should have the header ``\confidentialtext''
centered across both columns in the top margin.

Submissions must include line numbers in the left and right
margins, as demonstrated in the TACL submission-formatting
instructions pdf file, because the line numbering allows reviewers to be very
specific in their comments.\footnote{Authors using Word to prepare their
submissions can create the marginal line numbers by inserting text
boxes containing the line numbers.}
Note that the numbers on the ruler need not line up exactly with the text lines
of the paper. (Indeed, the line numbers generated by the recommended \LaTeX\
files typically do not correspond exactly to the text lines.)
\fi

The presence or absence of the ruler or header should not change the appearance
of any other content on the page.



\begin{table*}[t]
\centering
\begin{tabular}{p{7.8cm}@{\hskip .5cm}p{7.8cm}}
\multicolumn{1}{c}{{\bf Incorrect}} & \multicolumn{1}{c}{{\bf Correct}} \\  \hline
``\ex{(Cardie, 1992) employed learning.}'' &
``\ex{Cardie (1992) employed learning.}'' \\
{The problem}:  ``employed learning.'' is not a sentence.  & Create by
\verb+\citet{+\ldots\verb+}+  or \verb+\newcite{+\ldots\verb+}+. \\
\\  \hline
``\ex{The method of (Cardie, 1992) works.}'' &
``\ex{The method of Cardie (1992) works.}''  \\
{The problem}:  ``The method of was used.'' is not a sentence.  & Create as
above.\\ \\\hline
``\ex{Use the method of (Cardie, 1992).}'' &
``\ex{Use the method of Cardie (1992).}''  \\
{The problem}:  ``Use the method of.'' is not a sentence.  & Create as
above.\\ \\\hline
\ex{Related work exists Lee (1997).} & \ex{Related work exists (Lee,
1997).} \\
{The problem}:  ``Related work exists Lee.'' is not a sentence (unless one
is scolding a Lee). & Create by
\verb+\citep{+\ldots\verb+}+  or \verb+\cite{+\ldots\verb+}+. \\
\\  \hline
\end{tabular}
\caption{\label{tab:cite-commands} Examples of incorrect and correct citation
  format.  Also depicted are citation commands supported by the
  tacl2018.sty file, which is based on the natbib package and
  supports all natbib citation commands.
  The tacl2018.sty file also supports commands defined in previous ACL style
  files
  for compatibility.
  }
\end{table*}





\section{The First Page}
\label{ssec:first}

Center the title, which should be placed 2.5cm from the top of the page,
\iftaclpubformat
and author names and affiliations
\fi
across both columns of the first page. Long titles should be typed on two lines
without a blank line intervening.
\iftaclpubformat
After the title, include a blank line before the author block.
Do not use only initials for given names, although middle initials are allowed.
Do not put surnames in all capitals.\footnote{Correct: ``Lillian Lee'';
incorrect: ``Lillian LEE''.} Affiliations should include authors' email
addresses. Do not use footnotes for affiliations.
\else
Do not include the paper ID number assigned during the submission process.
\fi

\iftaclpubformat
\else
Although submissions should not include any author information, maintain space
for names and affiliations/addresses so that they will fit in the final
(camera-ready)
version.
\fi


Start the abstract at the beginning of the first
column, about 8 cm from the top of the page, with the centered header
``Abstract'' as specified in Table \ref{tab:font-table}.
The width of the abstract text
should be narrower than the width of the columns for the text in the body of the
paper by about 0.6cm on each side.

\section{Section headings}

Use numbered section headings (Arabic numerals) in order to facilitate cross
references. Number subsections with the section number and the subsection number
separated by a dot.



\section{Figures and Tables}

Place figures and tables in the paper near where they are first discussed.

Provide a caption for every illustration. Number each one
sequentially in the form:  ``Figure 1: Caption of the Figure.'' or ``Table 1:
Caption of the Table.''

Authors should ensure that tables and figures do not rely solely on color to
convey critical distinctions and are, in general,  accessible to the
color-blind.



\section{Citations and references}
\label{sec:cite}


\subsection{In-text citations}
\label{sec:in-text-cite}
Use correctly parenthesized author-date citations
(not numbers) in the text. To understand correct parenthesization, obey the
principle that \emph{a sentence containing parenthetical items should remain
grammatical when the parenthesized material is omitted.} Consult Table
\ref{tab:cite-commands} for usage examples.


\iftaclpubformat
\else
\subsection{Self-citations}
\label{sec:self-cite}

Citing one's own relevant prior work should be done,  but use the third
person instead of the first person, to preserve anonymity:
\begin{tabular}{l}
Correct: \ex{Zhang (2000) showed ...} \\
Correct: \ex{It has been shown (Zhang, 2000)...} \\
Incorrect: \ex{We (Zhang, 2000) showed ...} \\
Incorrect: \ex{We (Anonymous, 2000) showed ...}
\end{tabular}
\fi

\subsection{References}
\label{sec:references}
Gather the full set of references together under
the boldface heading ``References''. Arrange the references alphabetically
by first author's last/family name, rather than by order of occurrence in the
text.

References to peer-reviewed publications should be given in addition to or
instead of preprint versions. When giving a reference to a preprint, including
arXiv preprints, include the number.

List all authors of a given reference, even if there are dozens; do not
truncate the author list with an ``et al.''  Use full first/given names for
authors, not initials.  Include periods after middle initials.

Titles should have correct capitalization.  For example, change change
``lstm'' or ``Lstm'' to ``LSTM''.\footnote{If using BibTex, apply curly braces
within the title field to preserve intended capitalization.}   Capitalize the
first letter of the first word after a colon or similar punctuation mark.  For
book titles, capitalize the first letter of all main words.  See the
reference entry for \citet{Jurafsky+Martin:2009a} for an example.


We strongly encourage the following, but do not absolutely mandate them:
\begin{itemize}
\item Include DOIs.\footnote{The supplied \LaTeX\ files will
automatically add hyperlinks to the DOI when BibTeX or
BibLateX are invoked if the hyperref package is used and
the doi field is employed in the corresponding bib entries.
The DOI itself will not be separately printed out in that case.}
\item Include the version number when citing arXiv preprints, even if only one
version exists at the time of writing.
For example,\footnote{Bibtex entries for \citet{DBLP:journals/corr/cs-CL-0108005} and
\citet{DBLP:journals/corr/cs-CL-9905001} corresponding to the depicted output
can be found in the supplied sample file {\tt tacl.bib}.  We also cite
the peer-reviewed versions \cite{GOODMAN2001403,P99-1010}, as required.}
note the ``v1'' in the following.
\begin{quote}
Joshua Goodman.  2001.  A bit of progress in language modeling. {\it CoRR},
cs.CL/0108005v1.
\end{quote}
An alternative format is:
\begin{quote}
Rebecca Hwa. 1999. Supervised grammar induction using training data with limited constituent
information. {cs.CL/9905001}. Version 1.
\end{quote}
\end{itemize}

\section{Appendices} Appendices, if any, directly follow the text and the
references.  Recall from Section \ref{sec:length} that {\em appendices count
towards the page
limit.}


\iftaclpubformat

\section{Including acknowledgments}
Acknowledgments appear immediately before the references.  Do not number this
section.\footnote{In \LaTeX, one can use {\tt {\textbackslash}section*} instead
of {\tt {\textbackslash}section}.} If you found the reviewers' or Action
Editor's comments helpful, consider acknowledging them.
\else
\fi

\section{Contributors to this document}
\label{sec:contributors}

This document was adapted by Lillian Lee and Kristina Toutanova
from the instructions and files for ACL 2018, by Shay Cohen, Kevin Gimpel, and
Wei Lu. Those files were drawn from earlier *ACL proceedings, including those
for ACL 2017 by Dan Gildea and Min-Yen Kan, NAACL 2017 by Margaret Mitchell,
ACL 2012 by Maggie Li and Michael White, those from ACL 2010 by Jing-Shing
Chang and Philipp Koehn, those for ACL 2008 by Johanna D. Moore, Simone
Teufel, James Allan, and Sadaoki Furui, those for ACL 2005 by Hwee Tou Ng and
Kemal Oflazer, those for ACL 2002 by Eugene Charniak and Dekang Lin, and
earlier ACL and EACL formats,  which were written by several people,
including John Chen, Henry S. Thompson and Donald Walker. Additional elements
were taken from the formatting instructions of the {\em International Joint
Conference on Artificial   Intelligence} and the \emph{Conference on Computer
Vision and Pattern Recognition}.


\bibliography{tacl2018}
\bibliographystyle{acl_natbib}

\end{document}


